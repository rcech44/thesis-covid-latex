\chapter{Úvod}
\label{sec:Introduction}

Ke vzniku a~šíření onemocnění s~epidemickým nebo pandemickým potenciálem docházelo v~historii pravidelně. Prostřídalo se jich hodně ať už pravé neštovice, cholera nebo infekční žloutenka. V~současné době jsou stále přítomny epidemie a~pandemie, mezi nimiž se řadí například AIDS. Tento virus se již po mnoha desetiletí šíří po celé Africe a~stále více případů se vyskytuje v~Asii nebo Americe, přičemž způsobuje vysokou míru úmrtnosti \cite{pandemie-historie}. Na světě se ale objevilo nové koronavirové onemocnění covid-19, které způsobilo celosvětovou pandemii.

Pandemie znamená šíření infekčního onemocnění na celém světě, aniž by bylo omezeno na určité území nebo časové období. Epidemie na druhé straně znamená šíření onemocnění na určitém území a~v~omezeném časovém rámci \cite{pandemie-definice}. Jinak řečeno, epidemie se vyskytuje pouze v~určitém regionu, zatímco pandemie se šíří globálně. Některá infekční onemocnění, která mohou způsobit pandemii, jsou zvířecího původu a~jsou přenesena na člověka v~důsledku častých kontaktů se zvířaty, jako je chov, lov nebo i~obchod \cite{diseases}.

Dříve se na šíření epidemií podílely války, ale dnes má největší vliv na přenos nemoci snadné a~levné cestování. V~obou případech se jedná o~přesouvání velkého množství lidí v~prostoru, což vede k~šíření infekčních onemocnění. Informace o~onemocnění se začaly nanášet do map za účelem nalezení zdroje infekce a~lépe porozumět vlastnostem nemocí \cite{bednarkova-covid}.

V~dnešní době je poměrně snadné sledovat infekční onemocnění díky moderním technologiím v~oblasti zdravotnictví a~IT. V~případě nového koronavirového onemocnění existuje několik webových aplikací poskytující statistiky o~nemoci. Součástí těchto aplikací bývají i~grafické vizualizace území České republiky, jež umožňují uživatelům sledovat s~přehledem průběh nemoci, současný stav a~také nalézt kořeny jejího šíření. Aby byly tyto aplikace opravdu efektivní, musí být založeny na aktuálních a~ověřených datech, které jsou pro sledování epidemií nebo pandemií klíčové.

Cílem této práce je vytvořit aplikaci, která umožní lidem sledovat vývoj onemocnění~covid-19~na území České republiky se zohledněním na okresy. V~následující kapitole bude diskutováno samotné onemocnění a~existující aplikace společně s~dostupnými daty. Ve třetí kapitole budou analyzována možná řešení a~poté bude vybráno nejvhodnější řešení pro vytvoření aplikace s~analyzovanými požadavky. Poslední kapitola se zaměří na detailní popis nového řešení. Výsledná aplikace přinese veřejnosti nový pohled na pozorování covidu-19.

\endinput