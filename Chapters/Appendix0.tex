\chapter{Instalace a vytvoření projektu Djanga}

\begin{lstlisting}[style=bash,label=src:DjangoInstall,caption={Instalace a vytvoření projektu Djanga \cite{django-dokumentace}}]
# Zjištění verze Pythonu a instalace Pythonu verze 3.11 v Linuxu
$ python3 --version
$ sudo apt-get install python3.11

# Kontrola verze pip a instalace pip pro Python verze 3
$ pip --version
$ sudo apt-get install python3-pip

# Inicializace virtuálního prostředí 'my_venv' v současném adresáři
$ python3 -m venv my_venv

# Spuštění prostředí v systémech Linux/macOS
$ source my_venv/bin/activate

# Spuštění prostředí v systémech Windows
$ my_venv\Scripts\activate.bat

# Kontrola verze Djanga a instalace Djanga v Pythonu 3
$ python3 -m django --version
$ python3 -m pip install Django

# Instalace apscheduler v Pythonu 3
$ python3 -m pip install apscheduler



# Vytvoření nového projektu 'bp_project' v současném adresáři a vytvoření aplikace 'covid' v adresáři projektu
$ django-admin startproject bp_project
$ cd bp_project && python3 manage.py startapp covid
\end{lstlisting}

\endinput